\documentclass[../../main.tex]{subfiles}

\begin{document}
Encryption schemes are usually composed of a three tuple of $PPT$ algorithms, namely, key generation, encryption and decryption denoted by (KeyGen, Encrypt, Decrypt).

\begin{defn}
Let $M$ represent the message space, and let $\ell$ be the security parameter. A public key encryption scheme $\varepsilon$ on $M$ is a three tuple $(KeyGen, Encrypt, Decrypt) \in PPT$ satisfying the following functionalities:

\begin{itemize}
    \item $KeyGen_\varepsilon(1^\ell)$: The key generation algorithm takes as input the security parameter $1^\ell$ and outputs a public key, secret key pair $(pk,sk)$.
    \item $Encrypt_\varepsilon(pk,m)$: The encryption algorithm takes as input the public key $pk$ and a message $m \in M$ and outputs the corresponding ciphertext $c \in C$ where $C$ is the ciphertext space. 
    \item $Decrypt_\varepsilon(sk,c)$: The decryption algorithm takes as input the ciphertext $c \in C$ and secret key $sk$ and outputs the corresponding message $m \in M$.
\end{itemize}

\end{defn}

\noindent A public key encryption scheme must satisfy two requirements:

\begin{itemize}
    \item \textbf{Correctness}: $\forall m \in M$, the $Pr[(pk,sk) \leftarrow KeyGen(1^\ell): Dec(sk, Enc(pk,m))=m]=1$. In other words, a scheme is said to be correct if the decryption of an encrypted message under the corresponding keys returns the original message.
    \item \textbf{Semantic Security}: We define the notion of semantic security in terms of a game between a challenger $C$ and an Adversary $A$:\\
    \textit{\underline{Step 1}}: The challenger generates a random key pair $(pk,sk) \leftarrow KeyGen_\epsilon(1^\ell)$.
    
    \textit{\underline{Step 2}}: The adversary A sends $m_0$, $m_1$ to the challenger. 
    
    \textit{\underline{Step 3}}: The challenger generates a random bit, $b \leftarrow \{0,1\}$, and generates the ciphertext $c \leftarrow Enc(pk,m_b)$. 
    
    \textit{\underline{Step 4}}: The adversary outputs a $b' \in \{0,1\}$.

\noindent The advantage of the adversary A is
defined as follows:

\begin{equation*}
    Adv_A[\varepsilon] \myeq |Pr[b'=b]-1/2|
\end{equation*}

The encryption scheme $\varepsilon$ is said to be semantically secure if for all probabilistic polynomial
time adversaries $A$, the advantage $Adv_A[\varepsilon]$ is negligible.
\end{itemize}

\end{document}



















