\documentclass[../../main.tex]{subfiles}

\begin{document}
\noindent In order to complete our proof of correctness we need to ensure that the decryption of a ciphertext returns its corresponding plaintext. This is shown as follows. \\

\noindent In previous sections we have shown that the output ciphertext of encryption, homomorphic addition and homomorphic multiplication is of Type I:
\begin{equation*}
    c = r + (m + 2 r^*) \cdot (p-1)/2 + q \cdot p^2 
\end{equation*}

\noindent where $m$ is our original plaintext. Applying the decryption procedure, we get:
\begin{equation*}
    \begin{split}
        m &\leftarrow ((2c) \mod p) \mod 2 \\
          &\leftarrow (2(r + (m + 2r^*) \cdot (p-1)/2 + q \cdot p^2) \mod p) \mod 2 \qquad \qquad \qquad \qquad \\
          &\leftarrow ((2r + m \cdot p - m + 2r^*p - 2r^* + 2 q\cdot p^2) \mod p) \mod 2 \\
          &\leftarrow (-m \mod p) \mod 2 \\
          &\leftarrow (m \mod p) \mod 2 \\
    \end{split}
\end{equation*}

\noindent This concludes our proof of correctness.
\end{document}