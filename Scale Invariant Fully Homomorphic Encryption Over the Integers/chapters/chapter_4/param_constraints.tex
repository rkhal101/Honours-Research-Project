\documentclass[../../main.tex]{subfiles}

\begin{document}

The parameters used in this scheme are as follows. 

\begin{center}
\captionof{table}{List of Parameters [3]} 
\begin{tabular}{ |c|c| } 
 \hline
 \bf{Symbol} & \bf{Quantity} \\
 \hline
 $\lambda$ & Security parameter \\
 \hline
 $\gamma$ & Bit-length of integers in the public key \\
 \hline
 $\eta$ & Bit-length of private key \\ 
 \hline
 $\rho$ & Bit-length of the noise \\ 
 \hline
 $\tau$ & Number of integers in the public key \\
 \hline
 $\rho'$ & Secondary noise parameter used for encryption \\
 \hline 
 $L$ & Multiplicative depth of the circuit \\
 \hline
 $\Theta$ & Asymptotic tight bound \\
 \hline
 $\Omega$ & Asymptotic lower bound \\
 \hline
 $\omega$ & Weaker Asymptotic lower bound \\
 \hline
\end{tabular}
\end{center}

\noindent The parameters of this scheme must meet the following constraints $[7]$:

\begin{enumerate}
    \item $\rho = \Omega(\lambda)$, protects against brute force attacks on the noise,
    \item $\eta \geq \rho + \mathcal{O}(L \cdot log \lambda)$,
    \item $\gamma = \omega(\eta^2 \cdot log\lambda)$, thwarts lattice-based attacks,
    \item $\Theta^2 = \gamma \cdot \omega(log \lambda)$, avoids lattice attacks on the subset sum,
    \item $\tau \geq \gamma + 2\lambda$, allows use of Leftover Hash Lemma.
\end{enumerate}

\end{document}