\documentclass[../../main.tex]{subfiles}

\begin{document}

Fully Homomorphic Encryption over the integers was first introduced by Dijk, Gentry, Halevi and Vaikuntanathan (DGHV) $[6]$. As the name suggests, addition and multiplication operations are done over the integers with the semantic security of this scheme depending on the hardness of the Approximate GCD problem. This was the first scheme to implement FHE using simple mathematical operations. Following this scheme, several variant FHE implementations over the integers have been presented. 

In this section we'll describe one of the variant schemes introduced by Coron, Lepoint and Tibouchi known as Scale Invariant Fully Homomorphic Encryption over the Integers. This scheme provides the key feature that the single secret modulus grows linearly (as opposed to exponentially) with the multiplicative depth of the circuit when homomorphically evaluated $[7]$. Section 4.1 describes the parameters used in this scheme and lists the constraints necessary on these parameters. Section 4.2 describes the scale invariant FHE scheme. Section 4.3 provides a proof of correctness, and lastly, Section 4.4 reduces the security of this scheme to solving the Approximate GCD problem and thereby presents a proof of semantic security. 
\end{document}










