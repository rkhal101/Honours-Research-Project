\documentclass[../../main.tex]{subfiles}

\begin{document}
Alice owns an expensive jewelry store where she produces diamonds from raw material. Alice was doing great with her jewelery business until one day one of her employees took off with a quarter of her precious raw material. After intensive police work the culprit was brought to justice and the material was all restored to Alice; however this incident left Alice with trust issues when it comes to her employees. This got her thinking - is there a way of giving her employees the ability to process the raw material into diamonds without having access to the material itself?

After many sleepless nights, Alice came up with a plausible plan. She decides to use a transparent impenetrable glove box that has a lock on it. Alice keeps the key on her at all times. She uses the key to open the box whenever she needs to insert the raw material and locks the box after she is done. This way the workers can use the gloves to assemble the raw material into diamonds without having direct access to the raw material. Once the final product is produced, Alice opens the box with her key and takes out the assembled diamonds.

In the above analogy, the raw material represents our plaintext $m_1,...,m_t$. The transparent impenetrable glove box with the raw material inside of it represents the encryption of the plaintext. Opening (decrypting) the box can only be done using Alice's key. The gloves represent the homomorphism of the encryption scheme by allowing workers to perform manipulations on the raw data when it's in encrypted form (inside the box).

Of course, the above given analogy, as you might have noticed, is flawed. Nevertheless, it does give the reader an intuition for how homomorphic encryption works.
\end{document}