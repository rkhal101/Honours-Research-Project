\documentclass[../../main.tex]{subfiles}

\begin{document}
Let $Encrypt(pk,m \in \{0,1\})$ be defined as in Section 4.2. Then, for a random subset $S \subset \{1,...,\tau\}$, the ciphertext is encrypted as follows.

\begin{equation*}
    c = \left[m \cdot y + \sum_{i \in S}{x_i} \right]_{x_0}
\end{equation*}

\begin{align*}
    \begin{split}
        \implies c &= \left[m \cdot \left(y'+(p-1)/2\right) + \left(q_1 \cdot p^2 + r_1\right)\right]_{x_0} \text{ for some } q_1,r_1 \in \mathbb{Z}\\
        &= \left[m \cdot \left(q_2 \cdot p^2 + r_2 + \left(p-1\right)/2\right) + \left(q_1 \cdot p^2+r_1\right) \right]_{x_0} \text{ for some } q_1 \in \mathbb{Z} \cap [0,q_0), \\
        & \qquad\qquad\qquad\qquad\qquad\qquad\qquad\qquad\qquad\qquad\qquad\qquad r_1 \in \mathbb{Z} \cap (-2^\rho, 2^\rho) \\
        &= \left[m \cdot q_2 \cdot p^2 + m \cdot r_2 + m \cdot (p-1)/2 + (q_1 \cdot p^2 + r_1)\right]_{x_0} \\
        &= \left[(m \cdot r_2) + (m+(2r_1/(p-1))) \cdot (p-1)/2 +(m \cdot q_2 + q_1) \cdot p^2 \right]_{x_0}\\
        &= \left[r + (m+2r^*) \cdot (p-1)/2 + q \cdot p^2 \right]_{x_0}\\
    \end{split}
\end{align*}

\noindent where $r$ and $r^*$ are the two noises of the ciphertext. Coron, et al., call a ciphertext of this following form a Type I ciphertext.

\noindent Note that homomorphic addition and multiplication is done modulo $x_0$; however, we remove the modulus for simplicity without fear of it affecting our calculations.
\end{document}