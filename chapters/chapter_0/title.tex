\documentclass[../../main.tex]{subfiles}
\thispagestyle{empty}
\begin{document}
{\begingroup % Create the command for including the title page in the document
\centering % Center all text
\vspace*{\baselineskip} % White space at the top of the page

\rule{\textwidth}{1.6pt}\vspace*{-\baselineskip}\vspace*{2pt} % Thick horizontal line
\rule{\textwidth}{0.4pt}\\[\baselineskip] % Thin horizontal line

{\LARGE Scale Invariant Fully Homomorphic Encryption Over the Integers}% Title

\rule{\textwidth}{0.4pt}\vspace*{-\baselineskip}\vspace{3.2pt} % Thin horizontal line
\rule{\textwidth}{1.6pt}\\[\baselineskip] % Thick horizontal line

December 5, 2016\par % Location and year
\bigskip
\bigskip
\vspace*{2\baselineskip} % Whitespace between location/year and editors


\bigskip
\bigskip
\bigskip
PREPARED IN FULFILLMENT OF THE MATHEMATICS COURSE MAT5160 BY  \\[\baselineskip]
{\Large RANA KHALIL\par} % Editor list
\bigskip
\bigskip

\begin{comment}
    
{\itshape SUPERVISED BY\par} % Editor affiliation
\bigskip
{\Large CARLISLE ADAMS \& DANIEL FIORILLI\par} % Editor list

\vfill % Whitespace between editor names and publisher logo

\end{comment}

\vfill % Whitespace between editor names and publisher logo

\addcontentsline{toc}{section}{\protect\numberline{}Abstract}
\textbf{Abstract\\}

In this report we describe what it means for an encryption scheme to be fully homomorphic. This is illustrated through a simple scale invariant leveled fully homomorphic encryption scheme over the integers, that can be bootstrapped into a pure fully homomorphic encryption scheme. The semantic security of this encryption scheme is then analyzed and reduced to a computationally hard problem known as Approximate-GCD problem.
\endgroup}


\end{document}
